% !TEX program = xelatex
%%%%%%%%%%%%%%%%%
% This is an example CV created using altacv.cls (v1.1.3, 30 April 2017) written by
% LianTze Lim (liantze@gmail.com), based on the
% Cv created by BusinessInsider at http://www.businessinsider.my/a-sample-resume-for-marissa-mayer-2016-7/?r=US&IR=T
%
%% It may be distributed and/or modified under the
%% conditions of the LaTeX Project Public License, either version 1.3
%% of this license or (at your option) any later version.
%% The latest version of this license is in
%%    http://www.latex-project.org/lppl.txt
%% and version 1.3 or later is part of all distributions of LaTeX
%% version 2003/12/01 or later.
%%%%%%%%%%%%%%%%

%% If you want to use \orcid or the
%% academicons icons, add "academicons"
%% to the \documentclass options.
%% Then compile with XeLaTeX or LuaLaTeX.
% \documentclass[10pt,a4paper,academicons]{altacv}

%% Use the "normalphoto" option if you want a normal photo instead of cropped to a circle
% \documentclass[10pt,a4paper,normalphoto]{altacv}

\documentclass[10pt,a4paper]{altacv}

%% AltaCV uses the fontawesome and academicon fonts
%% and packages.
%% See texdoc.net/pkg/fontawecome and http://texdoc.net/pkg/academicons for full list of symbols.
%% When using the "academicons" option,
%% Compile with LuaLaTeX for best results. If you
%% want to use XeLaTeX, you may need to install
%% Academicons.ttf in your operating system's font %% folder.


% Change the page layout if you need to
\geometry{left=1cm,right=9cm,marginparwidth=6.8cm,marginparsep=1.2cm,top=1cm,bottom=1cm}

% Change the font if you want to.

% If using pdflatex:
\usepackage[utf8]{inputenc}
\usepackage[T1]{fontenc}
\usepackage[default]{lato}

% If using xelatex or lualatex:
%% \setmainfont{Lato}

% Change the colours if you want to
\definecolor{VividPurple}{HTML}{3E0097}
\definecolor{SlateGrey}{HTML}{2E2E2E}
\definecolor{LightGrey}{HTML}{666666}
\colorlet{heading}{VividPurple}
\colorlet{accent}{VividPurple}
\colorlet{emphasis}{SlateGrey}
\colorlet{body}{LightGrey}

% Change the bullets for itemize and rating marker
% for \cvskill if you want to
\renewcommand{\itemmarker}{{\small\textbullet}}
\renewcommand{\ratingmarker}{\faCircle}

%% sample.bib contains your publications
\addbibresource{sample.bib}

\begin{document}
\name{Luis Mayta}
\tagline{Site Reliability Engineer \& Proud Geek}
\photo{2.5cm}{avatar.jpg}
\personalinfo{%
  % Not all of these are required!
  % You can add your own with \printinfo{symbol}{detail}
  \email{slovacus@gmail.com}
  \phone{(+51) 959-196-850}
  \location{Lima, Peru}
  \homepage{luismayta.github.io}
  \twitter{@slovacus}
  \linkedin{linkedin.com/in/luismayta}
  \github{github.com/luismayta} % I'm just making this up though.
%   \orcid{orcid.org/0000-0000-0000-0000} % Obviously making this up too. If you want to use this field (and also other academicons symbols), add "academicons" option to \documentclass{altacv}
}

%% Make the header extend all the way to the right, if you want.
\begin{fullwidth}
\makecvheader
\end{fullwidth}

%% Provide the file name containing the sidebar contents as an optional parameter to \cvsection.
%% You can always just use \marginpar{...} if you do
%% not need to align the top of the contents to any
%% \cvsection title in the "main" bar.
\cvsection[page1sidebar]{Experience}

\cvevent{Bot Developer/SRE}{\textbf{ChatyDelivery}}{November 2017 -- Present}{Lima, Peru}

\divider

\cvevent{Cloud Engineer/SRE}{\textbf{Fashion Bag Peru}}{April 2017 -- Present}{Lima, Peru}

\divider

\cvevent{Cloud Architect}{\textbf{La Positiva Seguros}}{June 2017 -- November 2017}{Lima, Peru}

\divider

\cvevent{Software Architect}{\textbf{Ventorystack}}{January 2017 -- June 2017}{Lima, Peru}

\divider

\cvevent{Software Development Leader}{\textbf{Pimienta Digital}}{October 2014 -- July 2015}{Lima, Peru}

\divider

\cvevent{Technical Leader}{\textbf{Grupo El Comercio}}{April 2013 -- October 2014}{Lima, Peru}

\divider

\cvevent{Software Developer}{\textbf{Grupo El Comercio}}{March 2012 -- April 2013}{Lima, Peru}

\divider

\cvsection{A Day of My Life}

% Adapted from @Jake's answer from http://tex.stackexchange.com/a/82729/226
% \wheelchart{outer radius}{inner radius}{
% comma-separated list of value/text width/color/detail}
\wheelchart{1.5cm}{0.5cm}{%
  20/13em/accent!30/Sleeping \& dreaming about work,
  25/9em/accent!60/Resolving issues of code,
  10/12em/accent!40/Spending time with family,
  20/9em/accent/Research of tools and solutions,
  10/8em/accent!20/Relearning in automation,
  15/8em/accent!20/Resolving issues public in Github
}

\clearpage

\cvsection[page2sidebar]{Conferences}

\nocite{*}

%% \printbibliography[heading=pubtype,title={\printinfo{\faBook}{Books}},type=book]

%% \divider

%% \printbibliography[heading=pubtype,title={\printinfo{\faFileTextO}{Journal Articles}}, type=article]

%% \divider

%% \printbibliography[heading=pubtype,title={\printinfo{\faGroup}{Conference Proceedings}},type=inproceedings]

%% If the NEXT page doesn't start with a \cvsection but you'd
%% still like to add a sidebar, then use this command on THIS
%% page to add it. The optional argument lets you pull up the
%% sidebar a bit so that it looks aligned with the top of the
%% main column.
% \addnextpagesidebar[-1ex]{page3sidebar}


\end{document}
